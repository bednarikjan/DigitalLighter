\documentclass{report}

% cool tables
\usepackage{booktabs}
\newcommand{\ra}[1]{\renewcommand{\arraystretch}{#1}}

\newcommand{\colorA}{\cellcolor{green!100}}
\newcommand{\colorB}{\cellcolor{green!60}}
\newcommand{\colorC}{\cellcolor{yellow!75}}
\newcommand{\colorD}{\cellcolor{orange!90}}
\newcommand{\colorE}{\cellcolor{red!80}}
% images
\usepackage{graphicx}
\usepackage{tabularx}
\usepackage{rotating}
\usepackage[table]{xcolor} 
\usepackage{gantt}
\usepackage{hyperref}


% black square
\usepackage{amsmath}
\usepackage[parfill]{parskip}

%numbers
%\renewcommand\thesection{\arabic{section}}
\pagenumbering{arabic} 

\title{
	\hrule
    \normalsize \textsc{Customer Driven Project}\\
    \Huge Rock Concert Audience as a Screen\\[10pt]
    \normalsize Project Report\\[10pt]
    Netlight AS
    \hrule
    }  
\author{Agnethe Soraa,
Tomas Dohnalek,
Jan Bednarik,
Milos Jovac \\
\normalsize Project adviser: Anh Nguyen Duc}
\date{\today}

\begin{document}
\maketitle
<<<<<<< HEAD
\chapter*{Abstract}
\addcontentsline{toc}{chapter}{Abstract}
The purpose of this document is to give an insight into the details of the planning, research, design and implementation of the task given in the course TDT4290 - Customer Driven Project. 
The project aims to give the students experience with a real project, and with a real customer. 
This gives the students an opportunity to combine both theory and practice. 
The customer for our project is Netlight AS.  

Our project will be about researching and implementing image processing. 
Naturally this means we also have to solve problems regarding mapping of mocked units to locations as a function of time. 
The environment takes place at a rock concert, which means we also have to solve issues with timing and syncing between multiple independent units.

This is a proof-of-concept task.  
All the research done will be documented, and used to argue for and against the solutions. 
We will also argue for and against alternative solutions. Everything from the planning to the complete conclusion is described in this report. 
To be able to solve these problems we have to start by investigating relevant technologies, and how we can make this possible.
The conclusion of this study allows us to create a system which showcases the real potential of our solution.

\chapter*{Preface}
\addcontentsline{toc}{chapter}{Preface}
%\input{preface/preface.tex
\tableofcontents
\setcounter{page}{3}
\chapter*{Introduction}
\addcontentsline{toc}{chapter}{Introduction}
%\input{introduction/introduction.tex
\chapter*{Preliminary studies}
\addcontentsline{toc}{chapter}{Preliminary studies}
%\input{preliminaryStudies/preliminaryStudies.tex

\chapter*{Requirements}
\addcontentsline{toc}{chapter}{Requirements}
%\input{requirements/requirements.tex

\chapter*{Testplan}
\addcontentsline{toc}{chapter}{Testplan}
%\input{testplan/testplan.tex

\chapter*{Software Architecture}
\addcontentsline{toc}{chapter}{Software Architecture}
%\input{softwareArchitecture/softwareArchitecture.tex

\chapter*{Tools and strategy}z
\addcontentsline{toc}{chapter}{Tools and strategy}
%\input{toolsAndStrategy/toolsAndStrategy.tex

\chapter*{Sprint 0}
\addcontentsline{toc}{chapter}{Sprint 0}
In this chapter Sprint 0 will be described. It is written about what the team did during the first sprint, which also was a planning period for the team. At the end there will be evaluation of the  whole sprint, and answers to following questions: What went well? What could be improved? 

\section{Sprint planning}
The Sprint 0 is embraced as a preliminary sprint, where team can set up all necessary collaboration tools, equipment, prepare templates for meetings and mainly to acquaint team members with Scrum methodology. The original plan was to finish sprint 0 on 8th of September, but it have been decided to terminate it prematurely due to finishing sprint goals in shorter time than expected. Other reason for terminating the sprint was desire to start actual work on the product itself.

The user stories are listed in Table \ref{tab:sprint0stories}. Since software collaboration tool was used when sprint was already in progress, team did not manage to estimate the time needed to complete each story beforehand and thus the column \textbf{Est.} is left empty.

%\begin{longtable}{cX|cc}
%\textbf{ID} & \textbf{Description} & \textbf{Hours estim.} & \textbf{Hours spent} \\
%259 & \LaTeX template for minutes project plan & 5 & 4 \\
%259 & \LaTeX template for minutes project plan & 5 & 4 \\
%\end{longtable}

%\caption{User stories selected for Sprint 4.}
  \label{tab:sprint4stories}
 \def\arraystretch{1.25}
 
\begin{longtable}{ccXcc}

\toprule[0.5mm]
\multirow{2}{*}{\textbf{ID}} &
\multirow{2}{*}{\textbf{Ref.}} & \multirow{2}{*}{\textbf{Description}} & \multicolumn{2}{c}{\textbf{Hours}} \\
 					& & & \textbf{Est.} & \textbf{Sp.} \\
%\textbf{ID} 	& \textbf{Description} 	& \textbf{Est.} & \textbf{Sp.} \\
\midrule
\textbf{I4.1} 	& 	& {\bf As a server I need to link the devices' location with their ids.}	 &  52	& \textbf{48} \\

\textbf{I4.2} 	& 	& {\bf As a server I need to identifiy multiple clients from light.}		 &  19	& \textbf{18} \\

\textbf{I4.3} 	& 	& {\bf As a server I need to map all available devices to grid.} 			 & 22 & \textbf{18} \\	

\textbf{I4.4} 	& 	& {\bf As a server I need to play the whole media to the grid.} 			 & 37 & \textbf{34} \\
	
\midrule
		
				&& \textbf{SUM:}		&		130	& \textbf{136}
 \\																			
\bottomrule[0.5mm]
\end{longtable}


\section{Sprint goals}
\begin{itemize}
    \item Read compendium
    \item Find and agree suitable collaboration tools
    \item Start working on the Project plan
    \item Fill backlog with stories
    \item Assign roles to members of the team
    \item Assign a name to our project
\end{itemize}


\section{Testing}

Only the third party products were tested during the sprint 0. Team focused on testing the online team collaboration tools including Trello, described in Section \ref{subsec:TrelloToolDescription}, Gravity, described in Section \ref{subsec:GravityToolDescription} and TargetProcess, described in Section \ref{subsec:targetProcessToolDescription}. The outcomes of our research are described in Section \ref{sec:management_tools}.

\section{Occurring risks}

One of the major risks team came across were the limitations of the selected collaboration tools used for controlling the Scrum process. As team members were learning all the perks of using the Scrum methodology it was continuously discovered that neither Trello nor Gravity software tools suited team's needs and more time searching for another tool and setting it up again had to be spent. A few difficulties regarding the version control system Git prevents merging individual versions of the project plan document. This is considered as an risk so team spent significant amount of time solving these issues.

\section{Customer feedback}

The outcomes of preliminary studies have been presented to the customer. The customer was also walked through the collaboration tools and provided with the access to all of the team's resources so that the team and the customer would be able to collaborate more tightly. The customer was satisfied with the amount of the work done and approved the TargetProcess to become the team's main collaboration tool. Since the team managed to finish all of the selected user stories the customer agreed with the accelerated end of Sprint 0.

\section{Retrospective}
Since each complex software project requires that the team would go through the certain initiative tasks such as the assignment of the team roles, agreeing on the meetings, setting up the communication and collaboration tools etc., the Sprint 0 cannot be really thought of as a fully-fledged sprint. Team mainly focused on pre-studying the scrum methodology, communicating with the customer and the supervisor, clarifying the requirements and researching both the collaboration and development tools.

The working was started immediately after the first meeting with the customer was held. Besides the time spent on setting the tools we mainly focused on writing the project plan as this is a project where the amount of documentation is expected to be extensive. 

In order to reflect on the past sprint and to learn from the mistakes done it is necessary to answer two essential questions: "What went well" and "What could be improved".

\subsection{What went well}
Three online collaboration tools (Trello, Gravity and TargetProcess) were evaluated so that the suitable one could be find. The initiation tasks were finished earlier than expected and the Sprint 1 could be started earlier. All of the members' computers were set so that all of the tools would work. The team also got familiar with the concept of user stories, assigned a few of them to Sprint 0 and managed to finish all of them. All of the members were committed to spend the same amount of working hours.

\subsection{What could be improved}
During the first days the team was not operating efficiently due to the fact that team started to write the project plan without any clear specification of the document structure. Once the structure was agreed on the team members were able to work on the single tasks efficiently again without any need for unnecessary communication overhead and negotiations. Lack of experience in using the Scrum process resulted in imprecise usage of the user stories concept. The team should carry out additional research regarding the Scrum process in order to follow the specified guidelines properly. The team also did not measure the time spent on the different tasks. In order to reflect on the preliminary time estimation the time spent on working on the single tasks must be tracked. Late use of the proper collaboration tool resulted in inability to generate the burndown chart for Sprint 0.

\chapter*{Sprint 1}
\addcontentsline{toc}{chapter}{Sprint 1}
%\input{sprint1/sprint1.tex

\chapter*{Sprint 2}
\addcontentsline{toc}{chapter}{Sprint 2}
%\input{sprint2/sprint2.tex

\chapter*{Sprint 3}
\addcontentsline{toc}{chapter}{Sprint 3}
%\input{sprint3/sprint3.tex

\chapter*{Sprint 4}
\addcontentsline{toc}{chapter}{Sprint 4}
%\input{sprint4/sprint4.tex

\chapter*{Sprint 5}
\addcontentsline{toc}{chapter}{Sprint 5}
%\input{sprint5/sprint5.tex

\chapter*{Sprint 6}
\addcontentsline{toc}{chapter}{Sprint 6}
%\input{sprint6/sprint6.tex

\chapter*{Testing}
\addcontentsline{toc}{chapter}{Testing}
%\input{testing/testing.tex

\chapter*{Evaluation}
\addcontentsline{toc}{chapter}{Evaluation}
%\input{evaluation/evaluation.tex

\chapter*{Conclusion}
\addcontentsline{toc}{chapter}{Conclusion}
%\input{conclusion/conclusion.tex

\chapter*{References}
\addcontentsline{toc}{chapter}{References}
%\input{references/references.tex

\chapter*{Attachments}
\addcontentsline{toc}{chapter}{Attachments}
%\input{attachments/attachments.tex

\chapter*{Appendix}
\addcontentsline{toc}{chapter}{Appendix}
%\input{appendix/appendix.tex
=======
\section{Project customer}
Netlight AS is a consulting company engaged in IT and management. They operates throughout Europe with offices in Stockholm, Oslo, London, Munich and Helsinki. The company was founded at 1999 and employs to 500 employees.

\section{Project description}
The customer wants a product to make the audience as a screen on a rock concert. We decided to name "digital lighter". 
The audience members at a rock concert should be able to download a simple application to their cell phone, and register this through a simple GUI.
Behind the artist on stage there is a screen, with a simple camera on top. The camera is taking pictures of the audience. 
At special occasions the audience will be instructed, by the artist, to hold up their phones with the screen towards the stage.
This is similar to holding up a lighter like people did in the old days to create a special atmosphere at the concert. 
We want to digitalize this by giving the audience a chance to use their cell phones instead of lighters. 

On control a signal the application will fill the entire mobile screen with a single color.
The control signal can as an example say: "all pixels white". The signal will be specific for each application.
Each mobile will be a pixel in a larger picture, which will be presented on the big screen. 
What kind of picture the audience can create will depend on the number of people in the audience.   

As a motivation for the audience to hold up their phone, the camera on top of the screen will take pictures of the audience.
In that way the audience can see a reflection of them selves, and see what kind of picture they are creating together.  

\section{Project scope}

After third meeting with our customer we decided to lay out the scope of the project that is tangible and doable under 13 weeks. Customer idea although good and innovative has one flaw - it is a huge undertaking!
As we operate under certain limitations we explained in General Terms section, subsection Limitations, we agreed with customer on next terms:
 - Take project title as domain of work and not final product,
 - Scale down problem but attack all main problems of the domain,
 - Disregard scaling of product,
 - Disregard some of problems that are not important for final prototype, but talk with a customer before make this kind of decision.

\section{Product architecture}
Product can be logically divided into two sections -- client and server application.
Client side should be used by the audience whereas server side should be used by the concert manager.
Both kind of applications will communicate through network, there is no need of internet connection.

Each application can be divided into more detailed architecture; we will focus on that in next sprint.


\section{Measurement of Project Successes}
To measure success of our end-product we have to set up some criteria to be fulfilled. The product should pass all test-cases and function according to customer's requirements.

\section{Planned workload}
Compendium proposed week workload 25 person-hours per week. During our internal meeting we have decided that each member will spend 30 hours per week because our team consists only of 4 members. We agreed on fixed daily working hours so that we could distribute the workload through the whole semester. We will do daily stand-ups according to Scrum methodology.

\section{General Terms}
\subsection{Methodology}
After fast research and consultation with our mentor and customer(He have experience with CDP student group from last year), we decided to use SCRUM methodology for a 
few reasons. Requirements and scope of the undertaking were not that precisely defined at the time 
we had to decide on the methodology. Our approach therefore could not be founded on the sequential methodology as is "waterfall".
We wanted to have frequent meetings with our customer and involve him in development process. Sprints in SCRUM allow us just that - 
to have meeting with customer at the end of every sprint, and plan next one together. Sprints doesn't have to be the same length so 
we can better manage developing process and risks, and we will be able to have as much sprints as possible under time limit of 13 weeks. 
SCRUM will give us derivatives that we can enhance in increments and allow us to gradually reduce the risk and keep our customer informed 
about our progress. SCRUM is also a very popular approach in the software industry, so it is a good choice to learn it. 


\subsection{Tools selections}
For Scrum support and issue tracking we use Gravity Tool\footnote{\url{www.gravitydev.com}}. 
The tool is right now in Beta but it is free to use and have all features we needed from proposed AgileZen.
CHANGE(30.08.2013.) As Gravity missing some of the features we didn't know we need it was hard to procide in it.
New tool we established is Target Process 3 \footnote{\url{http://www.targetprocess.com/3/}}. Tool is also free and is not in Beta.

For collaboration on Minutes, Project Plan and other documents we use GitHub. This tool was proposed by our customer and it is popular free collaboration tool.
For document editing we agreed on LaTeX.
For group resources and links we use Facebook groups and for managing schedule we use Google Calendar.
 
\subsection{Limitations}
We are developing this project under a few technical, resource, time and knowledge limitations. 

Our biggest limitation is the image processing part. Half of the team has no experience with this, and the other half has little experience. Their experience is mostly theoretical information about the subject, and practical experience is preferred. We are aware of this limitation, and our plan is to learn by doing. We are going to start developing, and teach ourself while coding. We chose this approach because we do not want to spend more time than necessary doing research.

Another limitation is lack of experience with Mobile development within the development team. All of the team members have Android phones, and to be able to test our application, we have to develop an Android application. Only one team member have experience with this. 


If we are not scaling down the project, then we do not have all necessary resources to test the system. As an example we do not have a huge audience. Also  we do not have access to a big screen etc. This is also a limitation.  

As this course last for a 13 weeks, it is normal that we have to make some trade-offs.
This project is technically difficult, and there is a limited amount of time.  
  
\section{Schedule}

In general the sprints will last for two weeks. The only sprints that does not have that length, is the first and the last. This is because these two are more related to getting started, and finishing the project. The other sprints will have a more general approach. 

After each sprint we have a sprint review with the customer. The sprint reviews takes place every other Thursday, which means that the Wednesday before, the team has to prepare for the review. In each sprint review we both present and demonstrate what we have done so far. After the presentation we are doing a retrospective with the customer. It is important to evaluate our sprint delivery the customer. After the retrospective we will plan the next sprint, and now the customer can prioritize which user stories he wants us to work on next. The Friday after the sprint review, the team will do a detailed planning of the next sprint. 

On the Thursdays in between we have weekly meetings with the customer. It is important to have good communication with the customer. These meetings gives us opportunities to ask questions, and to make sure that we are on the right track.

\subsection{Phases}
\subsubsection{Sprint 0}
We have embraced Sprint 0 as a preliminary sprint, when we can set up all necessary collaboration tools, equipment, prepare templates for meetings and mainly to acquaint ourselves with Scrum methodology. The original plan was to finish sprint 0 on 8th of September, but we have decided to terminate it prematurely due to finishing sprint goals in shorter time than we had expected. Other reason for terminating the sprint was desire to start actually working on the product itself.

\paragraph{Sprint duration:} from 22nd of August until 1th of September
\paragraph{Sprint goals:}
\begin{itemize}
    \item Read compendium
    \item Find and agree suitable collaboration tools
    \item Start working the Project plan
    \item Fill backlog with stories
    \item Assign roles to members of the team
    \item Assign a name to our project
\end{itemize}



\paragraph{Sprint 1 (ends 15th of September)}
\paragraph{Sprint 2 (ends 29th of September)}
\paragraph{Sprint 3 (ends 13rd of October)}
\paragraph{Sprint 4 (ends 27st of October)}
\paragraph{Sprint 5 (ends 15th of November)}
\paragraph{Sprint 6 (ends 21th of November)}
\subsection{Gantt chart}
To visualize sprint's lengths and layout in time we created Gantt chart ~\ref{fig:gantt}.


\begin{figure}
\begin{center}
\label{fig:gantt}
\caption{Gantt Chart: Allocation of sprints into weeks.}
\begin{sideways}
\begin{gantt}{9}{14}
    \begin{ganttitle}
       \titleelement{Week}{14}
    \end{ganttitle}
    \begin{ganttitle}
       \titleelement{1}{1}
       \titleelement{2}{1}
       \titleelement{3}{1}
       \titleelement{4}{1}
       \titleelement{5}{1}
       \titleelement{6}{1}
       \titleelement{7}{1}
       \titleelement{8}{1}
       \titleelement{9}{1}
       \titleelement{10}{1}
       \titleelement{11}{1}
       \titleelement{12}{1}
       \titleelement{13}{1}
       \titleelement{14}{1}
    \end{ganttitle}
    \ganttbar{Sprint 0}{0.5}{1.5}
    \ganttbarcon{Sprint 1}{2}{2}
    \ganttbarcon{Sprint 2}{4}{2}
    \ganttbarcon{Sprint 3}{6}{2}
    \ganttbarcon{Sprint 4}{8}{2}
    \ganttbarcon{Sprint 5}{10}{2}
    \ganttbarcon{Sprint 6}{12}{1.5}
\end{gantt}
\end{sideways}
\end{center}
\end{figure}

>>>>>>> 51f36617481ea389eb9a81ba24cab30ba61b3fe7

\end{document}

%\subsection{General information}
%\subsection{Structure of report}
%\subsection{Project and project name}
%\subsection{Project purpose and concept}
%\subsection{Project goal}
%\subsection{Stakeholders}
%\subsubsection{Customer}
%\subsubsection{Customer contact}
%\subsubsection{Development team}
%\subsubsection{Advisor}
%\subsection{Project background}
%\section{Planning}
%\subsection{Project plan}
%\subsection{Methodology choice - Scrum}
%\subsection{Organization}
%\subsection{Risk Management}
%\subsection{Quality Assurance}
%\subsection{Measurement of project effects}
%\subsection{Duration and workload}
%\subsection{Gantt diagram}
%\subsubsection{Description}
%\subsection{Result schedule}
%\subsection{Roles}
%\subsection{Version Control}
%\subsection{Textual documentation}
%\subsection{Project background}
%\subsection{Source code}

%\subsection{Similar projects}
%\subsection{Market investigation}
%\subsection{Existing technologies and frameworks}
%\subsection{Evaluation of alternative solutions}
%\subsection{Outcome of research - Our decision}
%\subsection{Constraints}
%\subsection{Evaluation criteria}

%\subsection{Description/scope}
%\subsection{Definitions/general terms}
%\subsection{Business Requirements}
%\subsubsection{Functional}
%\subsubsection{Non-functional}
%\subsection{Use cases}
%\subsection{Product backlog}
%\subsection{Summary}

