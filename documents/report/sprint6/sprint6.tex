\section{Sprint planning}
It was formerly intended to devote the last sprint

Due to the fact the final version of the project must be submitted on the fixed date this sprint is shorter by two work days and the workload must be adjusted accordingly. The team has eight work days to its disposal which makes it for 140 person-hours for the whole team.

All the documentation related stories for sprint 6 are presented in Table \ref{tab:sprint6Documentationstories}. 
%\caption{User stories selected for Sprint 2.}
\def\arraystretch{1.25}
 
\begin{longtable}{ccXcc}
\label{tab:sprint2Documentationstories}\\[-6mm]
\caption{Documentation stories selected for sprint 2}\\[-4mm]
\toprule[0.5mm]
\multirow{2}{*}{\textbf{ID}} &
\multirow{2}{*}{\textbf{Ref.}} & \multirow{2}{*}{\textbf{Description}} & \multicolumn{2}{c}{\textbf{Hours}} \\
 					& & & \textbf{Est.} & \textbf{Sp.} \\
%\textbf{ID} 	& \textbf{Description} 	& \textbf{Est.} & \textbf{Sp.} \\
\midrule


\textbf{D2.1} 	& 
	\refwbs{wbs_documentation}{WBS 8.2}	& {\bf As a student I need to finish the pre-study chapter.} 									& 	12	& \textbf{ 16} \\

\textbf{D2.2} 	& 
	\refwbs{wbs_documentation}{WBS 8.2}	& {\bf As a student I need to finish the planning chapter.} 									& 	10	& \textbf{ 14} \\

\textbf{D2.3} 	&
	\refwbs{wbs_documentation}{WBS 8.2} 	& {\bf As a student I need to finish requirements chapter.} 									& 	30	& \textbf{ 26} \\

\textbf{D2.4} 	& 
	\refwbs{wbs_documentation}{WBS 8.2}  & {\bf As a student I need to finish the architecture chapter.} 								& 	24	& \textbf{ 12} \\

\textbf{D2.5} 	& 
	\refwbs{wbs_documentation}{WBS 8.2}	& {\bf As a student I need to finish sprint 1 chapter.} 										& 	12	& \textbf{ 16} \\

\textbf{D2.6} 	& 
	\refwbs{wbs_documentation}{WBS 8.2}	& {\bf As a student I need to work on the  sprint 2 chapter.} 									& 	16	& \textbf{ 18} \\
%ASK group about this:
%\textbf{360} 	& \refreq{}
%	& {\bf As a student I need to start on the architechture chapter.} 								& 	?	& \textbf{ ?} \\	

								
\hline
				&& \textbf{SUM:}		&		104	& \textbf{102}
 \\																			
\bottomrule[0.5mm]
\end{longtable}
 All the project management related stories for sprint 6 are presented in Table \ref{tab:sprint6storiesProcess}.
%\caption{User stories selected for Sprint 1.}
\label{tab:sprint1storiesProcess}
\def\arraystretch{1.25}
 
\begin{longtable}{ccXcc}

\toprule[0.5mm]
\multirow{2}{*}{\textbf{ID}} &
\multirow{2}{*}{\textbf{Ref.}} & \multirow{2}{*}{\textbf{Description}} & \multicolumn{2}{c}{\textbf{Hours}} \\
 					& & & \textbf{Est.} & \textbf{Sp.} \\
%\textbf{ID} 	& \textbf{Description} 									& \textbf{Est.} & \textbf{Sp.} \\
\midrule

% === Process ==========================
\textbf{326} 	& 
	& {\bf  As a student I have to track effort time} 	& 		16	& \textbf{16} \\
\textbf{345} 	& 
	& {\bf As a student I have attend the weekly meetings with the customer} 	
	& 	22	
	& \textbf{?} \\
		&& Preparation for demonstration	& 2 & ? \\
		&& Demonstration	& 6 & ? \\
		&& Writing minutes 	&  6 & ? \\	
		&& Customer meeting	&  6 & ? \\
		&& Writting minutes	&  2 & ? \\
		
\textbf{327} 	& 
	& {\bf As a student I have to attend the weekly meetings with the supervisor} 	
	& 	12	
	& \textbf{?} \\
		&& Meeting in week I	& 4 & ? \\
		&& Meeting in week II	& 4 & ? \\
		&& Writing minutes from week I 	&  2 & ? \\
		&& Writing minutes from week II	&  2 & ? \\	

\textbf{344} 	&& {\bf As a student I need to attend the team building.} 	& 		7	& \textbf{9} \\
		

\textbf{321} 	&& {\bf As a student I need to participate to lectures about team dynamics. } 	& 		32	& \textbf{25} \\
				&& Course of group dynamics Thu.	&  &  \\
				&& Summary of course and exchange learned.	&  &  \\				
				
\hline
				&& \textbf{SUM:}		&		164	& \textbf{?}
 \\																			
\bottomrule[0.5mm]
\end{longtable}


\subsection{Duration}
%This sprint is 2 weeks long. From 28th of October 2013 to 10th of November 2013. We agreed
%on the date of presentation and showing the running demo – on Thursday 7th of November 2013.
%Estimated velocity is 240 hours since we agreed on 30 working hours per person per week. 

\section{Sprint Goal}

\section{System Burndown}

\section{Architecture}
\section{Implementation}
\section{Testing}
\section{Occurring risks}
\section{Customer feedback}
- agreed that we fulfilled all the requirements

\section{Retrospective}
This section reflects on the past sprint. In order to learn from the mistakes done and thus to improve the workflow it is necessary to answer two essential questions: "What went well" and "What could be improved".

\subsection{What went well}
\subsection{What could be improved}
