The purpose of this document is to give an insight into the details of the planning, research, design and implementation of the task given in the course TDT4290 - Customer Driven Project. 
The project aims to give the students experience with a real project, and with a real customer. 
This gives the students an opportunity to combine both theory and practice. 
The customer for our project is Netlight AS.  

Our project will be about researching and implementing image processing. 
Naturally this means we also have to solve problems regarding mapping of mocked units to locations as a function of time. 
The environment takes place at a rock concert, which means we also have to solve issues with timing and syncing between multiple independent units.

This is a proof-of-concept task.  
All the research done will be documented, and used to argue for and against the solutions. 
We will also argue for and against alternative solutions. Everything from the planning to the complete conclusion is described in this report. 
To be able to solve these problems we have to start by investigating relevant technologies, and how we can make this possible.
The conclusion of this study allows us to create a system which showcases the real potential of our solution.

=====================

This document gives an insight into the details of planning, preliminary research, requirements, design, actual implementation and evaluation of prototype product named \emph{Digital Lighter}.

The resulting application gives rock concert audience an opportunity to take part during a concert in an active manner. 
All what is needed is a smart phone.
During the concert the audience is instructed to start a Digital Lighter application, connect to a server and raise their hands holding a smart phone's screen toward to the stage similar to holding up a lighter. 
Each screen then becomes a pixel on a giant screen and a video can be played on it.

This is a proof of concept task covering all main problems of the domain.



