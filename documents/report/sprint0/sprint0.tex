\section{Sprint planning}
The Sprint 0 is embraced as a preliminary sprint, where team can set up all necessary collaboration tools, equipment, prepare templates for meetings and mainly to acquaint team members with Scrum methodology. The original plan was to finish sprint 0 on 8th of September, but it have been decided to terminate it prematurely due to finishing sprint goals in shorter time than expected. Other reason for terminating the sprint was desire to start actual work on the product itself.

The user stories are listed in Table \ref{tab:sprint0stories}. Since software collaboration tool was used when sprint was already in progress team did not manage to estimate the time needed to complete each story beforehand and thus the column \textbf{Est.} is left empty.

%\begin{longtable}{cX|cc}
%\textbf{ID} & \textbf{Description} & \textbf{Hours estim.} & \textbf{Hours spent} \\
%259 & \LaTeX template for minutes project plan & 5 & 4 \\
%259 & \LaTeX template for minutes project plan & 5 & 4 \\
%\end{longtable}

%\caption{User stories selected for Sprint 4.}
  \label{tab:sprint4stories}
 \def\arraystretch{1.25}
 
\begin{longtable}{ccXcc}

\toprule[0.5mm]
\multirow{2}{*}{\textbf{ID}} &
\multirow{2}{*}{\textbf{Ref.}} & \multirow{2}{*}{\textbf{Description}} & \multicolumn{2}{c}{\textbf{Hours}} \\
 					& & & \textbf{Est.} & \textbf{Sp.} \\
%\textbf{ID} 	& \textbf{Description} 	& \textbf{Est.} & \textbf{Sp.} \\
\midrule
\textbf{I4.1} 	& 	& {\bf As a server I need to link the devices' location with their ids.}	 &  52	& \textbf{48} \\

\textbf{I4.2} 	& 	& {\bf As a server I need to identifiy multiple clients from light.}		 &  19	& \textbf{18} \\

\textbf{I4.3} 	& 	& {\bf As a server I need to map all available devices to grid.} 			 & 22 & \textbf{18} \\	

\textbf{I4.4} 	& 	& {\bf As a server I need to play the whole media to the grid.} 			 & 37 & \textbf{34} \\
	
\midrule
		
				&& \textbf{SUM:}		&		130	& \textbf{136}
 \\																			
\bottomrule[0.5mm]
\end{longtable}


\section{Sprint Goal}
\begin{itemize}
    \item Read compendium
    \item Find and agree suitable collaboration tools
    \item Start working the Project plan
    \item Fill backlog with stories
    \item Assign roles to members of the team
    \item Assign a name to our project
\end{itemize}

\section{System Burndown}
Since we managed to establish the proper collaboration tool Target Process 3 only during the sprint the software was not able to generate relevant burndown chart.
We at least tried to estimate how much time we spent working on each of the user stories listed in Table \ref{tab:sprint0stories}.

\section{Testing}

Only the third party products were tested during the sprint 0. Team focused on testing the online team collaboration tools including Trello \ref{subsec:TrelloToolDescription}, Gravity \ref{subsec:GravityToolDescription} and TargetProcess3 \ref{subsec:targetProcessToolDescription}. The outcomes of our research are describer in chapter \ref{sec:management_tools}.

\section{Occurring risks}

One of the more major risks team came across were the limitations of the selected collaboration tools used for controlling the Scrum process. As team members were learning all the perks of using the Scrum methodology it was continuously discovered that neither Trello nor Gravity software tools suited team's needs and more time searching for another tool and setting it up again had to be spent. A few difficulties regarding the version control system Git prevents merging individual versions of the project plan document. This might be considered as an risk which, and team spent significant amount of time solving these issues.

\section{Customer feedback}

The outcomes of preliminary studies have been presented to the customer. The customer was also walked through the collaboration tools and provided with the access to all of the team's resources so that the team and the customer would be able to collaborate more tightly. The customer was satisfied with the amount of the work done and approved the TargetProcess to become the team's main collaboration tool. Since the team managed to finish all of the selected user stories the customer agreed with the accelerated end of Sprint 0.

\section{Retrospective}
Since each complex software project requires that the team would go through the certain initiative tasks such as the assignment of the team roles, agreeing on the meetings, setting up the communication and collaboration tools etc., the Sprint 0 cannot be really thought of as a fully-fledged sprint. Team mainly focused on pre-studying the scrum methodology, communicating with the customer and the supervisor, clarifying the requirements and researching both the collaboration and development tools.

The working was started immediately after the first meeting with the customer was held. Besides the time spent on setting the tools we mainly focused on writing the project plan as this is a project where the amount of documentation is expected to be extensive. 

In order to reflect on the past sprint and to learn from the mistakes done it is necessary to answer two essential questions: "What went well" and "What could be improved".

\subsection{What went well}
Three online collaboration tools (Trello, Gravity and TargetProcess) were evaluated so that the suitable one could be find. The initiation tasks were finished earlier than expected and the Sprint 1 could be started earlier. All of the members' computers were set so that all of the tools would work. The team also got familiar with the concept of user stories, assigned a few of them to Sprint 0 and managed to finish all of them. All of the members were committed to spend the same amount of working hours.

\subsection{What could be improved}
During the first days the team was not operating efficiently due to the fact that team started to write the project plan without any clear specification of the document structure. Once the structure was agreed on the team members were able to work on the single tasks efficiently again without any need for unnecessary communication overhead and negotiations. Lack of experience in using the Scrum process resulted in imprecise usage of the user stories concept. The team should carry out additional research regarding the Scrum process in order to follow the specified guidelines properly. The team also did not measure the time spent on the different tasks. In order to reflect on the preliminary time estimation the time spent on working on the single tasks must be tracked. Late use of the proper collaboration tool resulted in inability to generate the burndown chart for Sprint 0.