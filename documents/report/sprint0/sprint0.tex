\section{Sprint planning}
The Sprint 0 is embraced as a preliminary sprint, where team can set up all necessary collaboration tools, equipment, prepare templates for meetings and mainly to acquaint team members with Scrum methodology. The original plan was to finish sprint 0 on 8th of September, but it have been decided to terminate it prematurely due to finishing sprint goals in shorter time than expected. Other reason for terminating the sprint was desire to start actual work on the product itself.

The user stories are listed in table \ref{tab:sprint0stories}. Since software collaboration tool was used when sprint was already in progress team did not manage to estimate the time needed to complete each story beforehand and thus the column \textbf{Est.} is left empty.

\subsection{Sprint 0 User-stories}

%\begin{longtable}{cX|cc}
%\textbf{ID} & \textbf{Description} & \textbf{Hours estim.} & \textbf{Hours spent} \\
%259 & \LaTeX template for minutes project plan & 5 & 4 \\
%259 & \LaTeX template for minutes project plan & 5 & 4 \\
%\end{longtable}

%\caption{User stories selected for Sprint 4.}
  \label{tab:sprint4stories}
 \def\arraystretch{1.25}
 
\begin{longtable}{ccXcc}

\toprule[0.5mm]
\multirow{2}{*}{\textbf{ID}} &
\multirow{2}{*}{\textbf{Ref.}} & \multirow{2}{*}{\textbf{Description}} & \multicolumn{2}{c}{\textbf{Hours}} \\
 					& & & \textbf{Est.} & \textbf{Sp.} \\
%\textbf{ID} 	& \textbf{Description} 	& \textbf{Est.} & \textbf{Sp.} \\
\midrule
\textbf{I4.1} 	& 	& {\bf As a server I need to link the devices' location with their ids.}	 &  52	& \textbf{48} \\

\textbf{I4.2} 	& 	& {\bf As a server I need to identifiy multiple clients from light.}		 &  19	& \textbf{18} \\

\textbf{I4.3} 	& 	& {\bf As a server I need to map all available devices to grid.} 			 & 22 & \textbf{18} \\	

\textbf{I4.4} 	& 	& {\bf As a server I need to play the whole media to the grid.} 			 & 37 & \textbf{34} \\
	
\midrule
		
				&& \textbf{SUM:}		&		130	& \textbf{136}
 \\																			
\bottomrule[0.5mm]
\end{longtable}


\section{Sprint Goal}
\begin{itemize}
    \item Read compendium
    \item Find and agree suitable collaboration tools
    \item Start working the Project plan
    \item Fill backlog with stories
    \item Assign roles to members of the team
    \item Assign a name to our project
\end{itemize}

\section{System Burndown}
Since we managed to establish the proper collaboration tool Target Process 3 only during the sprint the software was not able to generate relevant burndown chart.
We at least tried to estimate how much time we spent working on each of the user stories listed in table \ref{tab:sprint0stories}.

\section{Architecture} \label{txt:sprint0architecture}

Considering the well established team collaboration model "forming, storming, norming and performing" the sprint 0 focused mainly on the first three parts.
Thus no architecture was needed and this task was assigned to the sprint 1.

\section{Implementation}

As explained in section \ref{txt:sprint0architecture} the sprint 0 was not aimed at the implementation and this task was assigned to the sprint 1.

\section{Testing}

Only the third party products were tested during the sprint 0. Team focused on testing the online team collaboration tools including Trello \ref{TrelloToolDescription}, Gravity \ref{GravityToolDescription} and TargetProcess3 \ref{targetProcessToolDescription}. The outcomes of our research are describer in chapter \ref{txt:development tools}.

\section{Occurring risks}

One of the more major risks team came across were the limitations of the selected collaboration tools used for controlling the Scrum process. As team members were learning all the perks of using the Scrum methodology it was continuously discovered that neither Trello nor Gravity software tools suited team's needs and more time searching for another tool and setting it up again had to be spent. A few difficulties regarding the version control system Git prevents merging individual versions of the project plan document. This might be considered as an risk which, and team spent significant amount of time solving these issues.

\section{Retrospective}

Since each complex software project requires that the team would go through the certain initiative tasks such as the assignment of the team roles, agreeing on the meetings, setting up the communication and collaboration tools etc., the Sprint 0 cannot be really thought of as a fully-fledged sprint. Team mainly focused on pre-studying the scrum methodology, communicating with the customer and the supervisor, clarifying the requirements and researching both the collaboration and development tools.

It feels like team managed to do a lot of work due to the fact working was started immediately after the first meeting with the customer was held. Besides the time spent on setting the tools we mainly focused on writng the project plan as this is a project where the amount of documentation is expected to be extensive. Though during the first days team was not working particularly efficient due to the fact that team started to write the project plan without any clear specification of the document structure. Once structure was agreed, team dare to claim they were performing very well which resulted in finishing all of the initiative tasks before the formerly set deadline.

The outcomes of preliminary studies have been presented to the customer and team also walked him through the collaboration tools and provided him with the access to all of the resources so that they would be able to collaborate more tightly. The customer was satisfied with the work done which resulted in his approval of the accelerated end of Sprint 0.

\subsection{Pros}
\begin{itemize}
\item A few online collaboration tools were evaluated so that suitable one could be find.
\item The initiation tasks were finished earlier than expected and the Sprint 1 could be started earlier.
\item All of the members' computers were set so that all of the tools work.
\item Team got familiar with the concept of user stories, assigned a few of them to Sprint 0 and managed to finish all of them.
\item All of the members were committed to spend the same amount of working hours.
\end{itemize}

\subsection{Cons}
\begin{itemize}
\item Lack of experience in using the scrum process.
\item Missing document structure resulted in loosing time figuring out what to work on.
\item The time spent on different tasks was not measured.
\item Late use of the proper collaboration tool resulted in inability to generate the burndown chart.
\end{itemize}
