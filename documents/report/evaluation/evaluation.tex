The evaluation is one of the final phases of this project. In this chapter, we will evaluate different aspects of the projects. 
Here it will be discussed why and how the outcome ended up being what it is today.
This includes how the team worked together, why it gave the result it did, the cooperation with the customer, and how it was working with an overseeing force. Furthermore,we will discuss the issues met during the project, and how the process we used worked for us.
We have chosen to split the evaluation into three major parts, group evaluation, project evaluation and technology evaluation. 
 
 
This is the final phase of this project, the evaluation. Here it will be discussed why and how the outcome
ended up being what it is today.  
The technology evaluation will be a fairly straightforward discussion and evaluation of our experiences with the technology we used in this project.
%_______________________________________________________________________________
\section{Group evaluation}
The group evaluation will discuss the social aspects of the project. 
We will discuss the team dynamics, our goals, how we used our role assignment, risk assessment, the advisor the customer, and other issues regarding group evaluation in general.
The project evaluation will be a discussion of the work process and technical problems we encountered.

\subsection{Team dynamics}

\subsection{Communication}
The communication was done through frequent physical meetings, and if someone had other obligations and they were not able to be precent, then we kept in contact though Facebook. This was a suitable solution for us since the team members were available online nearly all the time. 
This made coordination an easier task, since other members could be consulted at any time. 
Furthermore, our small group size allowed us to get well known with each other, and this made the communication better.

Key words for further writing

working hours and people not showing up in time cuz they did something late at night and then figures it was okay not to show up, then the rest of the group did not show up either.


+pysical meetings
+facebook
+stand ups, could have done more of those
+meetings, did not have weekently, but we asked for them if we neededto

-Not alwas CC in emails to the whole group
-Some members did not get the information when needed 


write something about conflicts resolution?

All in all there wherent personal issues, friends, just discussions regaring solutions to the project. 

\subsection{Goals} 
\subsection{Language Barriers}

\subsection{Work distribution}

The work distribution was very dynamic. We agreed on what needed to be done, and then group members had to take individual responsibility to perform work on the tasks they were comfortable with.
We could have distributed the work more evenly through out the project. 
It was a little stessful towars the ending of the project. Especially since we had to prioritize making the demo-video in the 5th sprint. Then we had a little trouble with the demo-video, and we had to spend more hours than we first thought. Because of this writing the documentation was a lower priority, and we fell a little behind.  
  

\subsubsection{Effort and estimation}
Throughout the project, time restrictions has been an issue for the group. Most of the 
members in the group have had other group projects during the semester, making it hard to 


\subsection{Role assignment}
\subsection{Risk evaluation}
\subsection{Customer} 
\subsection{Advisor}
%_______________________________________________________________________________
\section{Project Evaluation}
This section will start with an evaluation of the planning phase and our preliminary studies.  
Then we will look at our use of the scrum method, and how we had to modify it for our project, before moving on to a discussion of our conduction of meetings. 
\subsection{Planning}
\subsection{Preliminary Studies}
\subsection{Scrum}
\subsection{Meetings-Summary}
\subsection{Course feedback}
\subsection{Testing}
\subsection{Time usage}
%_______________________________________________________________________________
\section{Technologies and tools evaluation}
\subsection{Git}
The team adopted Git as its version control system since early phase as described in section \ref{subsec:git}.
Most of the members had experience with some other VCS such as Subversion and therefore the idea of versioning was not new.
On the other hand, three of four members of the team had no experience with Git itself.

Even though Git is a powerful tool with a lot of features, the main reason of using was to share code.
Since development was rather linear, branching was used rarely.
On the other hand, support of releases was used quite often \footnote{\url{https://github.com/dohnto/DigitalLighter/releases}}.
This was very comfortable for creating a final report, when the team could easily see exactly what features were included in each prototype and also it was possible to se the GUI of product in particular phase.

Of course, troubles during merging occurred.
Since merging conflicts can be sometimes demanding and usually requires a lot of knowledge about merging code, sometimes it was impossible for one person to solve it.
These conflicts were in most cases solved by discussion with all interested members.

After all the experience with Git is rather positive. 
It is always good to be able to work with tools, that a lot of companies use in real development.


\subsection{TestFlight}
TestFlight service, described in section \ref{subsec:testflight}, was adopted because of customer proposal.
The service was used in early phases of development for customer comfort when testing the product.

Since the beginning of implementation, some non-reproducible errors occurred in client application.
After some testing, the team suspected TestFlight of those problems.
TestFlight for Android is relatively new service \footnote{\url{http://blog.testflightapp.com/post/49971420302/android}} and there it was decided to discard it from the implementation.
After some time, the team has discovered, that the issues were caused by Android system itself (REF NEEDED), but the TestFlight was never used again, because the demonstration were performed via videos.
 
Therefore it is difficult to evaluate tool, that has been used only for a short period of time.

\subsection{TargetProcess3}
\subsection{OpenCV}
OpenCV is an open source computer vision library briefly described in section \ref{subsec:image_processing_library}.
It is a very useful library, which saved the team a lot of time instead of implementing image processing parts themselves
 -- whole light detection module was just enhanced demonstration example of OpenCV functionality.

On the other hand, some problems with this library occurred.
OpenCV Java API is a relatively new and therefore some implementation is missing.
One example of this was described in section \ref{sec:sprint3_implementation}.
This raised a lot of problems, which resulted into discarding of user story.

OpenCV Java API can be therefore counted as a immature product but with a high potential in future, provided OpenCV developers will focus more on this API.
At this moment, new relase 2.4.7 if OpenCV is 44 days late according to the plan\footnote{\url{http://code.opencv.org/projects/opencv/roadmap}} with still 32 opened bugs or new features.
\subsection{Android}
\subsection{Technical issues}
