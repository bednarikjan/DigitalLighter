This chapter is devoted to describing the outcomes of the preliminary research focusing on the similar already existing projects and technologies we could utilize.

- possibly sth like:\\
----\\
In this chapter preliminary studies will be presented, starting from current situation in the 
Cloud Systems. After that a WS-agreement will be presented and the explanation for the 
team choices will be given. \\
-----

Since this is a prototype project, a
considerable time is put into this part of the project, to assure the team makes good choices when it

This chapter is ment to outline the preliminary study of our project. This includes what our technologies
aims to achieve and how we will use them to achieve this.

\section{Similar projects}

We were not able to find any other service that would use exactly the same technical solution as compared to our product. 
Nevertheless a few similar entertainment services that aim to amuse the music concert audience already exist. 
These services basically encourage the audience to use either their smartphones or other device as the source of light that helps to create impressive and colorful show.

\subsection{Wham City Lights}

\subsubsection{Description}
The Wham City Lights is the mobile application developed by the one year old Baltimore based start-up \footnote{\url{http://whamcitylights.com/}}. 
As far as the Digital Lighter is concerned the Wham City Light service seems to be the closest solution as it works with the users' smartphones and uses them as a source of the light and even the sound.

Users attending the given concert only need to download and install the free application, start it and then hold their smartphones in the air.
The application then controls the color displayed on the smartphone screen, the camera flashes and the sound going out of the speakers and it creates the spatial soundscapes and lighting designs.
What is more the application do not require the Internet connection as the instructions are modulated into the ultrasonic inaudible signal that is being continuously transmitted during the performance.

Using this innovative technique all of the smartphones can be effectively synchronized so that the  changing lights and imagery on the phone follows and complete the music performance.
Nevertheless the so called light show must be pre-programmed and this system cannot determine the location of the mobile devices.

The demonstration of the real world performance of the application and the whole concept for that matter can be seen in the official Wham City Lights marketing video\footnote{\url{http://www.youtube.com/watch?v=faJ1Av5kBCE}}.

\subsubsection{Impact}
The Wham City Lights application or its derivatives have been so far used on relatively many music events of the greater importance where several thousands of people were present and used the application. To list a few:
\begin{itemize}
\item America's Got Talent 2013
\item CMA Music Festival 2013
\item Intel sales conference 2013
\item Billboard Music Awards 2013
\item etc.
\end{itemize}

\subsubsection{Availability}
The application is available for the devices using Android or iOS operating systems and it can be downloaded from the Google Play\footnote{https://play.google.com/store/apps/details?id=com.whamcitylights} and App Store\footnote{https://itunes.apple.com/us/app/wham-city-lights/id580034697?mt=8} respectively. The developer nevertheless offers the customers the concert specific applications with relevant user interface built-in. As for the light show itself, programming part can be done either by the customer or the developer.

\subsubsection{Relationship to our project}
Like our project the Wham City Lights application build upon users' smartphones that are remotely controlled in order to display intended imagery. 

\paragraph{Advantages}
The devices attending the light show do not need the Internet connection and the displayed imagery is well synchronized with the music.

\paragraph{Disadvantages}
The location of the mobile devices cannot be determined and the different processing speed of the devices causes incorrect synchronization.


\subsection{Xylobands}

\subsubsection{Description}
So called Xyloband\footnote{http://xylobands.com/} is the invention of the company RB Concepts Limited. The device itself consists of the plastic bracelet that includes the LED diodes of the different colors and the microcontroller. 
The bracelets are controlled by the radio signal being broadcast from the stage and as a result they change their colors, flashes and in general create colorful imagery.

\subsubsection{Impact}
Xylobands received great fame thanks to the well known British rock band ColdPlay which used them during their 2012 tour.

\subsubsection{Availability}
The bracelets can be ordered on the official Xylobands website.

\subsubsection{Relationship to our project}
the Xylobands product is somewhat different from the Digital Lighter as it does not utilize the users' mobile devices. 
The similarity can be found in the way the bracelets are synchronized using wireless signal. 

\paragraph{Advantages}
The users do not need to bring their own mobile device and all of the audience have the possibility to become the part of the light show.

\paragraph{Disadvantages}
The location of the single participants cannot be determined.

%\section{Existing technologies and frameworks}
%\section{Evaluation of alternative solutions}

\section{Development technologies}

For Scrum support and issue tracking we use Gravity Tool\footnote{\url{www.gravitydev.com}}. 
The tool is right now in Beta but it is free to use and have all features we needed from proposed AgileZen.
CHANGE(30.08.2013.) As Gravity missing some of the features we didn't know we need it was hard to procide in it.
New tool we established is Target Process 3 \footnote{\url{http://www.targetprocess.com/3/}}. Tool is also free and is not in Beta.

For collaboration on Minutes, Project Plan and other documents we use GitHub. This tool was proposed by our customer and it is popular free collaboration tool.
For document editing we agreed on LaTeX.
For group resources and links we use Facebook groups and for managing schedule we use Google Calendar.



\section{Outcome of research}
