\section{Project plan}
\section{Methodology choice - Scrum}
After fast research and consultation with our mentor and customer(He have experience with CDP student group from last year), we decided to use SCRUM methodology for a few reasons. Requirements and scope of the undertaking were not that precisely defined at the time we had to decide on the methodology. Our approach therefore could not be founded on the sequential methodology as is "waterfall". We wanted to have frequent meetings with our customer and involve him in development process. Sprints in SCRUM allow us just that - 
to have meeting with customer at the end of every sprint, and plan next one together. Sprints doesn't have to be the same length so we can better manage developing process and risks, and we will be able to have as much sprints as possible under time limit of 13 weeks. SCRUM will give us derivatives that we can enhance in increments and allow us to gradually reduce the risk and keep our customer informed about our progress. SCRUM is also a very popular approach in the software industry, so it is a good choice to learn it.
\section{Organization}
\section{Risk Management}
In the table \ref{tab:risks} the consequence and possibility in a number between 1-10. The risk, or the riskfactor, is the consequence multiplied with the possibility. The risks in this table is very obvious ones. 

Other risks we can see from analyzing the skill table. If for example the only two persons on the team who are familiar with image processing is away, then this will be a risk. There is a great possibility that this kind of task will take longer time then planned for. 

Also if some of the rows in the skill table all were painted red, which would imply that the team had experience with this, then this would be a risk. In such a case, we would have to talk to our customer and consider scaling down this task. 
\ref{tab:risks}

\begin{table*}
    \caption{Handling risks}
    \label{tab:risks}
    
    \centering \ra{1.3}
    \vspace{2mm} %\rotatebox{90}{
    \begin{tabularx}{500pt}{XlllXX}
    \toprule
        Event & Consequence & Possibility & Risk  & Reactive Measures & Proactive Measures \\
    \midrule
Someone gets sick & 4     & 5     & 20    & Other people do more work.  & Free weekends \\
Coding problems & 4     & 7     & 28    & Talk to supervisor \& Guru office & preparing for the task \\
Testing problems & 4     & 4     & 16    & Talk to customer about reformulating requirements & Double check requirements with customer \\
Implementing things we are not supposed to & 7     & 6     & 42    & Try to adopt functionality or start all over & Don't do anything that is not in backlog and keep good communication with customer \\
Dead end with technologies & 8     & 8     & 64    & Talk to supervisor \& Guru office & Do thoroughly research \\
Unrealistic time estimate & 7     & 8     & 56    & Work overtime  & Planing poker \\
Frequent changes in requirements specification & 6     & 3     & 18    & Renegotiate with a customer & Try no to change finished modules and keep weekly meetings with the customer \\
Customer too ambitious & 9     & 5     & 45    & Renegotiate with a customer & Keep customer informed about what  to expect \\
Hardware problems & 9     & 3     & 27    & Obtain a new one & Keep your devices updated \\
\bottomrule
\end{tabularx}
\end{table*}

\section{Quality Assurance}
\section{Measurement of project effects}
\section{Duration and workload}
\section{Gantt diagram}
You can see the Gantt chart in figure \ref{fig:gantt}.
\begin{figure}
    \begin{center}
        \label{fig:gantt}
        \caption{Gantt Chart: Allocation of sprints into weeks.}
        \begin{sideways}
            \begin{gantt}{9}{14}
                \begin{ganttitle}
                   \titleelement{Week}{14}
                \end{ganttitle}
                \begin{ganttitle}
                   \titleelement{1}{1}
                   \titleelement{2}{1}
                   \titleelement{3}{1}
                   \titleelement{4}{1}
                   \titleelement{5}{1}
                   \titleelement{6}{1}
                   \titleelement{7}{1}
                   \titleelement{8}{1}
                   \titleelement{9}{1}
                   \titleelement{10}{1}
                   \titleelement{11}{1}
                   \titleelement{12}{1}
                   \titleelement{13}{1}
                   \titleelement{14}{1}
                \end{ganttitle}
                \ganttbar{Sprint 0}{0.5}{1.5}
                \ganttbarcon{Sprint 1}{2}{2}
                \ganttbarcon{Sprint 2}{4}{2}
                \ganttbarcon{Sprint 3}{6}{2}
                \ganttbarcon{Sprint 4}{8}{2}
                \ganttbarcon{Sprint 5}{10}{2}
                \ganttbarcon{Sprint 6}{12}{1.5}
            \end{gantt}
        \end{sideways}
    \end{center}
\end{figure}

\subsection{Description ?}
\subsection{Result schedule ?}
\section{Roles}
\section{Version Control}
\section{Textual documentation}
